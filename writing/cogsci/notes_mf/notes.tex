\documentclass[fleqn,reqno,10pt]{article}

\RequirePackage{txfonts} % for strict implication symbols
% \RequirePackage{amsmath}            % Formeln
% \RequirePackage{amsfonts}           % Fonts for Formulas
% \RequirePackage{amssymb}
\RequirePackage[ngerman,english]{babel}
\RequirePackage[utf8]{inputenc}
\RequirePackage[T1]{fontenc} 
\RequirePackage[style=authoryear-comp, % Citation marks as [Jef65]
            natbib=true,      % Natbib-style cite macros \citeauthor &c.
            hyperref=true,    % Cites in pdf are links to bib
                              % (hyperref conf.)
            maxnames=3,       % truncate name lists if more than 2
                              % names appear
            doi=false,        % no doi's
            url=false,        % no url's
            sortcites=false,  % do NOT sort cites in the style of the
                              % bibliography 
            %backref=true     % insert backrefs in reference section
            ]{biblatex}
\bibliography{MyRefGlobal}
\usepackage[margin = 2cm]{geometry}

\title{Notes}
\author{}
\date{}

\begin{document}


If you hear ``Jones is happy'' you understand that Jones' happiness ---a \emph{gradable
  property}--- exceeds a certain threshold
\citep{KennedyMcNally2005:Scale-Structure,Kennedy2007:Vagueness-and-G}. Where exactly this
threshold lies is a matter of context, in particular prior expectations about usual degrees of
happiness of a person like Jones, but can be derived formally by considerations of
goal-oriented cooperative language use
\citep{LassiterGoodman2015:Adjectival-vagu,QingFranke2014:Gradable-Adject}. The model of
\citet{LassiterGoodman2015:Adjectival-vagu}, for example, captures a listener's probabilistic
reasoning about which thresholds would likely explain a speaker's utterance if the speaker
strives to maximize information flow about the true degree of Jones' happiness. Here, we try to
extend this line of inquiry to also incorporate compositional expressions involving various
forms of negation: how happy or sad is an agent who is reported to be \emph{not happy},
\emph{not sad}, \emph{unhappy} or \emph{not unhappy}? We propose a probabilistic
speaker-listener pragmatic reasoning model in the Rational Speech Act tradition
\citep{FrankGoodman2012:Predicting-Prag,FrankeJager2015:Probabilistic-p,GoodmanFrank2016:Pragmatic-Langu}. Our
model introduces elements of lexical uncertainty
\citep{BergenLevy2012:Thats-what-she-,BergenLevy2014:Pragmatic-Reaso} in order to model a
listener's uncertainty about how to interpret overt negation markers.

Negation is the semantic operation of forming an opposite, but there are several kinds of
semantic opposition \citep{Horn1989:A-Natural-Histo,sep-negation}. A \emph{contrary
  opposition}, such as between \emph{happy} and \emph{sad}, is one where both predicates cannot
be true at the same time, but can be false at the same time. A \emph{contradictory opposition},
such as between \emph{pure} and \emph{impure}, is one where truth of one predicate entails
falsity of the other. In other words, gradable terms that express contrary opposites allow for
a neutral middle ground, unlike contradictory opposites. Contrary opposition is thus logically
stronger than contradictory opposition, but frequently natural language expressions that appear
to express contradictory opposition are pragmatically strengthened to convey contrary
opposition instead. For example, today's English adverb \emph{never} derives from Old English
\emph{n\={\ae}fre}, a combination of \emph{ne} (not) and \emph{\={\ae}fre} (\emph{ever})
literally meaning \emph{not ever}, but strengthened to mean \emph{never}.

\emph{Lexical antonyms} are pairs like \emph{happy} and \emph{sad} or \emph{polite} and
\emph{rude}. Intuitively, lexical antonyms normally express contrary opposition: Jones can be
neither happy nor sad without logical contradiction.

We distinguish two markers of negation in English, \emph{adverbial negation} (e.g., \emph{not
  happy}, \emph{not polite}) and \emph{affixal negation} (e.g., \emph{unhappy},
\emph{impolite}). In principle, both types of negation markers could map onto either type of
semantic opposition relation, contrary or contradictory opposition. If $Hx$ expresses that $x$
is happy, we denote a contradictory opposition using standard bivalent negation $\neg
Hx$.
Contrary opposition is formed by a different kind of negation, which we denote as
$\tilde{H}x$. The latter is a predicate-forming operation that is not iterable
\citep{Horn1989:A-Natural-Histo,sep-negation}. So while it makes sense to iterate $\neg \neg
Hx$, it is impossible to iterated contrary negation. It is, however possible, to have $\neg
\tilde{H}x$. But it is not possible to have $\neg$ in the scope of $\tilde{H}x$. 

This means that listeners may in principle be uncertain about which of several logical readings
a speaker has in mind when she uses a negated gradable predicate. In particular, \emph{not
  happy} and \emph{unhappy} can both be construed as either $\neg Hx$ or $\tilde{H}x$. Given
the constraints mentioned above, the phrase \emph{not unhappy} can receive two interpretations,
namely $\neg Hx$ and $\neg \tilde{H}x$. 

In the probabilistic model that combines pragmatic reasoning about the meaning of gradable
adjectives with reasoning about the meaning of negation markers, we will assume that contrary
negation $\tilde{H}x$ leads to a new independent threshold (with the right directionality). In
other words, we leave it to rational pragmatic reasoning to decide optimal meaning assignments
to predicates, including those formed from predicate-forming contrary negation. 

This leads to the \emph{uncertain parse model} \dots




\printbibliography[heading=bibintoc]

\end{document}
