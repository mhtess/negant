% 
% Annual Cognitive Science Conference
% Sample LaTeX Paper -- Proceedings Format
% 

% Original : Ashwin Ram (ashwin@cc.gatech.edu)       04/01/1994
% Modified : Johanna Moore (jmoore@cs.pitt.edu)      03/17/1995
% Modified : David Noelle (noelle@ucsd.edu)          03/15/1996
% Modified : Pat Langley (langley@cs.stanford.edu)   01/26/1997
% Latex2e corrections by Ramin Charles Nakisa        01/28/1997 
% Modified : Tina Eliassi-Rad (eliassi@cs.wisc.edu)  01/31/1998
% Modified : Trisha Yannuzzi (trisha@ircs.upenn.edu) 12/28/1999 (in process)
% Modified : Mary Ellen Foster (M.E.Foster@ed.ac.uk) 12/11/2000
% Modified : Ken Forbus                              01/23/2004
% Modified : Eli M. Silk (esilk@pitt.edu)            05/24/2005
% Modified : Niels Taatgen (taatgen@cmu.edu)         10/24/2006
% Modified : David Noelle (dnoelle@ucmerced.edu)     11/19/2014

%% Change "letterpaper" in the following line to "a4paper" if you must.

\documentclass[10pt,letterpaper]{article}


\newcommand*\diff{\mathop{}\!\mathrm{d}}
\newcommand{\denote}[1]{\mbox{ $[\![ #1 ]\!]$}}

\usepackage{cogsci}
\usepackage{pslatex}
\usepackage{apacite}


\title{``Not unreasonable'': Carving a semantic scale with alternative utterances}
 
\author{{\large \bf Michael Henry Tessler (mtessler@stanford.edu)} \\
  Department of Psychology, Stanford University 
  \AND {\large \bf Michael Franke (mchfranke@gmail.com)} \\
  Department of Linguistics, University of T\"{u}bingen}


\begin{document}

\maketitle


\begin{abstract}

Language provides multiple, seemingly identical ways to convey opposite meanings. A speaker can  alter  the morphology of  a  word  (e.g.,  ``reasonable''  -->  ``unreasonable'')  or  employ  a  negating  modifier  (e.g.,  ``not reasonable''). However, doing both---producing a negated antonym (e.g., ``not  unreasonable'')?intuitively does not get you back to where you started (i.e., ``reasonable''). Instead, the four constructions that result from employing both types of negation appear to carve up the underlying semantic scale, producing an ordering (on levels of reasonableness): ``unreasonable?--``not reasonable?--``not unreasonable?--``reasonable? \cite{Krifka2007:Negated-antonyms}. We show how this basic ordering falls out of a minimal elaboration of the gradable adjectives model of \citeA{Lassiter2013}. We then empirically investigate this ordering, showing that the ordering is sensitive to the utterances the speaker?s alternative utterances for antonym pairs defined by morphological negation (``reasonable?/``unreasonable?) but not for antonym pairs defined with distinct lexical items (``tall? / ``short?). I?ll conclude by showing how an RSA model with uncertainty over the speaker?s set of alternative utterances produces this behavior. This work suggests that, consistent with psycholinguistic evidence \cite<e.g.,>{DegenTanenhaus2015:Processing-Scal}, different sets of alternative utterances may themselves compete and be resolved through pragmatic reasoning.

\textbf{Keywords:} 
semantics; pragmatics; Rational Speech Act; Bayesian cognitive model; adjectives
\end{abstract}


\section{Introduction}

The entire contribution of a proceedings paper (including figures,
references, and anything else) can be no longer than six pages.

The text of the paper should be formatted in two columns with an
overall width of 7 inches (17.8 cm) and length of 9.25 inches (23.5
cm), with 0.25 inches between the columns. Leave two line spaces
between the last author listed and the text of the paper. The left
margin should be 0.75 inches and the top margin should be 1 inch.
\textbf{The right and bottom margins will depend on whether you use
  U.S. letter or A4 paper, so you must be sure to measure the width of
  the printed text.} Use 10~point Times Roman with 12~point vertical
spacing, unless otherwise specified.

The title should be in 14~point, bold, and centered. The title should
be formatted with initial caps (the first letter of content words
capitalized and the rest lower case). Each author's name should appear
on a separate line, 11~point bold, and centered, with the author's
email address in parentheses. Under each author's name list the
author's affiliation and postal address in ordinary 10~point type.

Indent the first line of each paragraph by 1/8~inch (except for the
first paragraph of a new section). Do not add extra vertical space
between paragraphs.


\section{Computational model}

Gradable adjectives (e.g., ``tall'', ``reasonable'') convey the value along a scale (e.g., \emph{height}, \emph{reasonableness}) for some referent $x$ is above or below some contextual standard: $\denote{tall} = \emph{height}(x) > \theta$.

\section{Experiment 1: Lexical antonyms}
\subsection{Methods}
\subsubsection{Participants}

We recruited N participants from Amazon?s Mechanical Turk. Participants were restricted to those with U.S. IP addresses and who had at least a 95\% work approval rating. The experiment took on average N minutes and participants were compensated \$N for their work.

\subsubsection{Materials}

We used antonym-pairs of gradable adjectives.
The adjectives were all individual-level predicates describing persistent properties of an individual (e.g., ``tall'', not ``drunk''). 
Adjective-pairs (positive P, negative N) appeared in one of five forms: ``P'', ``not P'', ``N'', ``not N'', and ``neither P nor N''. 

\subsubsection{Procedure}

On each trial, participants read a statement introducing a character using a gradable adjective in one of the five structural forms (e.g., ``Greg is \emph{adj}'').
Participants were asked to interpret the statement by answering: ``How P do you think Greg is?'', where P was the positive-form adjective (e.g., ``How polite do you think Greg is?'').
The response variable was a 101-pt slider bar with endpoints labeled: ``100\% N'' and ``100\% P''.
Participants completed 30 trials in randomized order.

\subsection{Results}

\section{Experiment 2: Morphological antonyms}


\subsection{Methods}
\subsubsection{Participants}
\subsubsection{Materials}
\subsubsection{Procedure}
\subsection{Results}


\section{Experiment 3: Salient alternatives}

\subsection{Methods}
\subsubsection{Participants}
\subsubsection{Materials}
\subsubsection{Procedure}
\subsection{Results}


\section{Acknowledgments}

Place acknowledgments (including funding information) in a section at
the end of the paper.


\section{References Instructions}

Follow the APA Publication Manual for citation format, both within the
text and in the reference list, with the following exceptions: (a) do
not cite the page numbers of any book, including chapters in edited
volumes; (b) use the same format for unpublished references as for
published ones. Alphabetize references by the surnames of the authors,
with single author entries preceding multiple author entries. Order
references by the same authors by the year of publication, with the
earliest first.

Use a first level section heading, ``{\bf References}'', as shown
below. Use a hanging indent style, with the first line of the
reference flush against the left margin and subsequent lines indented
by 1/8~inch. Below are example references for a conference paper, book
chapter, journal article, dissertation, book, technical report, and
edited volume, respectively.

\bibliographystyle{apacite}

\setlength{\bibleftmargin}{.125in}
\setlength{\bibindent}{-\bibleftmargin}

\bibliography{negant}


\end{document}
